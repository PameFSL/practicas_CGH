\documentclass[12pt, a4paper]{article}
\usepackage{geometry}
\usepackage{multicol}
\usepackage{multirow}
\usepackage[table]{xcolor}
\usepackage{enumerate}
\usepackage{listings}
\usepackage{minted}
\geometry{margin=1in}

\usepackage[utf8]{inputenc}
\usepackage[T1]{fontenc}
\usepackage{graphicx}
\usepackage{ragged2e}

\usepackage{minted}
\usepackage{rotating}
\usepackage{hyperref}
\usepackage{float}
\usepackage{pstricks}
\usepackage{makecell}
\usepackage{fancyhdr}
\pagestyle{fancy}
\fancyhead{}
\fancyfoot{}

\fancyhead[l]{Laboratorio de Computación Gráfica e Interacción humano-computadora}
\fancyhead[r]{Grupo: 02}
\fancyfoot[r]{\thepage}

\usepackage{amsmath}
\usepackage{tikz}
\usepackage{longtable}
\usetikzlibrary{automata, positioning, arrows,shapes.misc, shapes.arrows, chains,matrix,positioning,% wg. " of "
	scopes,%
	decorations.pathmorphing,% /pgf/decoration/random steps | erste Graphik
	shadows}
\renewcommand{\refname}{Referencias}
\renewcommand{\contentsname}{Contenidos}
\renewcommand{\figurename}{Figura}
\renewcommand{\partname}{Parte}

\begin{document}
	
	\begin{titlepage}
		
		\newcommand{\HRule}{\rule{\linewidth}{0.5mm}} % Define comando para lineas horizontales
		
		\centering % Centra todo en la pagina
		
			%----------------------------------------------------------------------------------------
		%   LOGO SECTION
		%----------------------------------------------------------------------------------------
		
		\includegraphics[scale = 0.235]{img/logo.png} % Logo
		\hspace{4cm}
		\includegraphics[scale = .32]{img/fi.png}\\[.65cm] % Logo
	%----------------------------------------------------------------------------------------
		%   ENCABEZADO
		%----------------------------------------------------------------------------------------
		
		\textsc{\large \bfseries UNIVERSIDAD NACIONAL AUTÓNOMA DE MÉXICO}\\[.5cm] % Nombre de universidad
		\textsc{\large FACULTAD DE INGENIERÍA}\\[0.5cm] % Division
		\textsc{\large Ingeniería en Computación}\\[1.4 cm] % Carrera
		
		%----------------------------------------------------------------------------------------
		%   TITLE SECTION
		%----------------------------------------------------------------------------------------
		{\large LABORATORIO DE COMPUTACIÓN GRÁFICA\\ E INTERACCIÓN HUMANO COMPUTADORA}\\[.4 cm] % Materia
		% llllllllllllllllllllllllllllllllllllll
		\textsc{\large Grupo: 02}\\[1.5 cm]
		\textsc{\large Práctica Número 1}\\[1.5 cm]
		{\large Alumna: Pamela Salgado Fernández Pamela}\\[.3 cm]
		{\large Número de Cuenta: 313236505}\\[.3 cm]
		{\large Email: pame501@yahoo.com.mx}\\[1.6 cm]
		\raggedleft 
		{\large Semestre 2019-2}\\[.3 cm]
		\textsc{\large Grupo de teoría: 1}\\[.3 cm]
		{\large Fecha de entrega límite: 11/02/2019}\\[.5 cm] 

			%----------------------------------------------------------------------------------------
		
		\vfill % Llenar el resto de la página con espacio en blanco
		 
		
	\end{titlepage}
	\tableofcontents
	\newpage
	\noindent
	
\section{Desarrollo de la práctica}
\justify
En esta práctica, aprendimos a crear un nuevo proyecto así como a realizar la configuración de OpenGL ya que para poder compilar los programas tenemos que configurar el proyecto para que vincule los archivos de librería de openGl, glew y glfw.

\subsection{Creación del proyecto}

\begin{enumerate}
	\item Abrir Visual Studio \\[.5 cm]
	\centering 
	\includegraphics[scale = .38]{img/ab_proyecto_01.jpg}\\[.35cm] % Imagen1
	Figura 1 \\[.5cm]
	\raggedright 
	\item Crear un nuevo proyecto.
	\item Elegir la opción de proyecto vacío.
	
	\raggedright 
	\item Seleccionar la ubicación en donde queremos guardar nuestro proyecto.\\[.1 cm]
	\centering 
	\includegraphics[scale = .65]{img/ab_proyecto_03.jpg}\\[.25cm] % Imagen1
	Figura 2 \\[.35cm]
	
\end{enumerate}

\subsection{Agregar archivo .cpp al proyecto}
\begin{enumerate}
	\raggedright 
	\item En la parte derecha se podrá visualizar el explorador de soluciones.
	\item Dar click derecho en archivos de origen.
	\item Seleccionar "agregar" --> Elemento existente \\[.25cm]
	\centering 
	\includegraphics[scale = .65]{img/ab_proyecto_05.jpg}\\[.25cm] % Imagen1
	Figura 3 \\[.35cm]
	\raggedright 
	\item Después nos desplegará una ventana, copiamos la dirección que nos da por defecto y en esa misma ruta, copiamos los archivos de la carpeta que nos proporcionó el profesor.\\[.25cm] 
	\centering 
	\includegraphics[scale = .8]{img/ab_proyecto_08.jpg}\\[.25cm] % Imagen1
	Figura 4 \\[.5cm]
	\raggedright 	
	\item En la ventana que nos desplegó en el punto numero 4, seleccionamos el archivo "mainbase.cpp"\\[.25cm] 
	\centering 
	\includegraphics[scale = .6]{img/ab_proyecto_09.jpg}\\[.25cm] % Imagen1
	Figura 5 \\[.5cm]
	\end{enumerate}
\subsection{Vincular los archivos de librería}	
	\begin{enumerate}
	\raggedright 	
	\item En la parte derecha en la parte del explorador de soluciones, damos click derecho en donde esta el nombre de nuestro proyecto, de las opciones que nos despliega, seleccionamos "propiedades" \\[.25cm] 
	\centering 
	\includegraphics[scale = .6]{img/ab_proyecto_10.jpg}\\[.25cm] % Imagen1
	Figura 6 \\[.5cm]
	\raggedright 	
	\item Dentro de las propiedades de nuestro proyecto, modificamos lo siguiente:
		\begin{enumerate}
			\item En la parte de propiedades de configuración
			\\Damos click izquierdo en C/C++ --> General \\
			Y en Directorios de inclusión adicionales, agregamos {\itshape include} \\[.25cm] 
			\centering 
			\includegraphics[scale = .6]{img/ab_proyecto_11.jpg}\\[.25cm] % Imagen1
			Figura 7 \\[.5cm]
			\raggedright 
			\item En la parte de propiedades de configuración
			\\Damos click izquierdo en Vinculador --> General \\
			Y en Directorios de bibliotecas adicionales, agregamos {\itshape lib} \\[.25cm] 
			\centering 
			\includegraphics[scale = .6]{img/ab_proyecto_12.jpg}\\[.25cm] % Imagen1
			Figura 8 \\[.5cm]
			
			\raggedright 
			\item En la parte de propiedades de configuración
			\\Damos click izquierdo en Vinculador --> Entrada \\
			Y en Dependencias adicionales, agregamos {\itshape opengl32.lib; glew32.lib; glfw3.lib} \\[.25cm] 
			\centering 
			\includegraphics[scale = .6]{img/ab_proyecto_13.jpg}\\[.25cm] % Imagen1
			Figura 9 \\[.5cm]
		\end{enumerate}
		\item Dar click izquierdo en aplicar
		\item Y finalmente en {\itshape aceptar } \\[.5cm]
	\end{enumerate}
\newpage	
			\raggedright 
\subsection{Ejecución del programa}	
		\subsubsection{En laboratorio}
			\centering 
			\includegraphics[scale = .4]{img/im.png}\\[.25cm] % Imagen1
			Figura 10 \\[.5cm]
		\raggedright 
		\subsubsection{En casa}
		\centering 
		\includegraphics[scale = .42]{img/casa.png}\\[.25cm] % Imagen1
			Figura 11 \\[.5cm]
\newpage
 \raggedright 	
\section{Problemas presentados}
\justify
Durante la realización de la práctica en mi casa, como ya habían pasado algunos días desde que había hecho la práctica en laboratorio, no recordaba muy bien en que parte se copiaban los archivos de openGL 3.3 así que los copié en la carpeta en donde estaba el archivo.sin del proyecto (los cuales no iban ahí) y al momento de ejecutar el código, me arrojaba los siguientes errores: \\[.5cm]

\centering 
		\includegraphics[scale = .58]{img/errores.jpg}\\[.25cm] % Imagen1
			Figura 12 \\[.5cm]

 \justify	
Yo creí que el error era que había configurado mal las propiedades del proyecto y gracias a que un amigo me hizo el favor de revisar mi configuración comprendí que el error estaba en que los archivos no se copiaban en esa carpeta y una vez que los copié en la carpeta correcta, el programa compiló correctamente.
\\[.5cm]
\section{Comentarios}
\justify
Al principio se me complicó mucho la práctica en el laboratorio, ya que no pude llegar temprano a la clase y ya habían realizado varios puntos, la maquina que me tocó, se encontraba apagada y tanto para encender como para abrir Visual Studio se tardó un poco y por si fuera poco, olvidé cual era la contraseña de mi cuenta, así que no podía entrar a Visual Studio y mis compañeros seguían avanzando con la práctica. \\[0.3 cm]
Cuando por fin recordé mi contraseña, mis compañeros básicamente ya habían terminado la practica, por suerte, había anotado algunos de los pasos que habían realizado y gracias a la ayuda de mis compañeros y la del profesor, la configuración fue realizada de forma correcta y    el código funcionó sin problemas. \\[0.3 cm]
Me parece que esta práctica fue muy importante, ya que en ella aprendimos a crear un proyecto nuevo y realizar las configuraciones necesarias para poder utilizar las librerías necesarias, algo que es muy importante, ya que esto es algo que vamos a realizar muchas veces durante el semestre y es necesario hacerlo bien para poder realizar nuestros programas utilizando al máximo las herramientas que nos proporciona openGL.\\[0.3 cm]
Me gustaría que cuando realicemos las prácticas, las computadoras que usamos ya se encuentren prendidas, para que así tengamos más tiempo para realizar las prácticas y si surgen dudas, tener tiempo para poder abordarlas.\\[0.3 cm]

Además también fui capaz de verificar si las características que había investigado de mi tarjeta de gráficos eran correctas, y parece ser que si, ya que los resultados obtenidos en el programa y los de mi investigación son iguales.\\[0.3 cm]

\centering 
		\includegraphics[scale = .58]{img/inf.jpg}\\[.25cm] % Imagen1
			Figura 13: Información obtenida de la pagina:  \url https://www.geektopia.es/es/product/intel/intel-hd-graphics-630/   \\[.5cm]

\end{document}